\documentclass[12pt]{article}
\usepackage{enumerate}
\usepackage[hmargin=2.0cm,vmargin=1cm]{geometry}
\usepackage[utf8]{inputenc}

\title{\begin{LARGE}
{HW1: UNIX - Scripts}
\end{LARGE}}
\begin{document}

\maketitle

Esta tarea se debe entregar antes del final de la clase a través de Sicua plus. Un solo script debe resolver todos los puntos requeridos, y el nombre del script debe llevar su nombre y apellido en el formato \verb"NombreApellido_HW1.sh". \\

En el archivo hipparcos.csv se encuentra información astronómica de estrellas pertenecientes a 
la vía láctea, nuestro ejercicio consiste en procesar este archivo usando algunas de las herramientas de la terminal (wget, tar, wc, grep, echo, $>$, $>>$). El archivo se puede descargar de la dirección: \verb"https://dl.dropboxusercontent.com/u/9988568/hipparcos.tar.gz"\\ 
\\
1. Descargar y descomprimir el archivo hipparcos.tar.gz (15 puntos) \\
2. ¿Cuántas estrellas hay en el catálogo? (15 puntos) \\
3. ¿Cuánto pesa este catálogo en bytes? (10 puntos) \\
4. ¿Cuántas estrellas son del mismo tipo espectral que el Sol \textbf{G2V}? (20 puntos) \\
5. Hacer un catálogo que contenga los rótulos de las columnas, y las filas correspondientes a las estrellas de tipo G2V, este catálogo debe tener el nombre \verb"EstrellasG2V.csv" (20 puntos)\\
6. ¿Cuánto pesa el nuevo catálogo en bytes? (10 puntos)\\
7. Comprimir el catálogo parcial quedando con el nombre \verb"EstrellasG2V.tar.gz". (10 puntos)

\end{document}