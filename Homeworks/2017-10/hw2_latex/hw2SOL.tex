\documentclass{article}
\usepackage{amsmath, amssymb}
\author{Luis Alberto Guti\'errez L\'opez}
\date{\today}
\title{Tarea 2 - Herramientas computacionales}

\begin{document}
\maketitle

\section{Horario semanal}

\begin{tabular}{| c | c | c | c | c | c |}
\hline
Hora& Lunes & Martes & Miercoles & Jueves & Viernes \\
\hline
6:30  &  & Lab. Metodos S1 & Herramientas S1 &  &  \\
\hline
7:00  &  & Lab. Metodos S1 & Herramientas S1 &  &  \\
\hline
7:30  &  & Lab. Metodos S1 & Herramientas S1 &  &  \\
\hline
8:00  &  & Herramientas S3 &  &  &  \\
\hline
8:30  &  & Herramientas S3 &  &  &  \\
\hline
9:00  &  & Herramientas S3 &  &  &  \\
\hline
9:30  &  &  &  &  &  \\
\hline
10:00 &  &  &  &  &  \\
\hline
10:30 &  &  &  &  &  \\
\hline
11:00 &  &  &  &  &  \\
\hline
11:30 &  &  &  &  &  \\
\hline
12:00 &  &  &  &  &  \\
\hline
12:30 & Metodos S1 &  & Metodos S1 &  &  \\
\hline
1:00  & Metodos S1 &  & Metodos S1 &  &  \\
\hline
1:30  & Metodos S1 &  & Metodos S1 &  &  \\
\hline
2:00  &  & Metodos S2 &  & Metodos S2 &  \\
\hline
2:30  &  & Metodos S2 &  & Metodos S2 &  \\
\hline
3:00  &  & Metodos S2 &  & Metodos S2 &  \\
\hline
3:30  &  &  &  &  &  \\
\hline
4:00  &  &  &  &  &  \\
\hline
4:30  &  &  &  &  &  \\
\hline
5:00  &  & Seminario &  &  &  \\
\hline
5:30  &  & Seminario &  &  &  \\
\hline
6:00  &  & Seminario &  &  &  \\
\hline
6:30  &  &  &  &  &  \\
\hline
\end{tabular}

\section{Ecuaciones de Maxwell}
	\begin{equation*}
		\vec{\nabla}\cdot\vec{E} = \frac{\rho}{\epsilon_0}
	\end{equation*}
	\begin{equation*}
		\vec{\nabla}\cdot\vec{B} = 0
	\end{equation*}
	\begin{equation*}
		\vec{\nabla}\times\vec{E} = -\frac{\partial\vec{B}}{\partial t}
	\end{equation*}
	\begin{equation*}
		\vec{\nabla}\times\vec{B} = \mu_0\vec{J}+\mu_0\epsilon_0\frac{\partial\vec{E}}{\partial t}
	\end{equation*}

\end{document}
